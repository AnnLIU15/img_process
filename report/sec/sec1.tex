\section{引言}
在户外的视觉智能系统中,常常会遇到可见性的问题,光照引起的问题会导致视觉智能系统性能极具下降,比如在夜晚中使用基于白天光照条件设计、训练出来的视觉系统,其表现会急剧下降甚至无法正常使用。除了夜晚这类光照强度问题外,不同的天气也会不同程度导致场景画面的能见度降低,从而影响了户外智能系统的性能。

本文复现的算法针对的就是有雾场景下的图像处理,大气粒子散射和吸收从物体表面反射的光。这种散射和吸收在恶劣天气下变得严重,导致辐照度的大小不正确,图像与视频会因为雾的存在导致性能急剧下降,图像出现退化的情况。相机接收到一小部分直接从物体表面反射的光,以及大量被大气粒子反射的光。因此,物体的对比度和可见性会减弱。户外视觉智能系统在我们的生活生产中遍地可见,如物体检测和识别、视觉监控、交通监控、智能交通。但是,由于图像和视频的退化,视觉智能系统表现不佳。因此,视觉智能系统需要一种方法解决雾对图像的影响。早期的去雾方法基于附加信息。这些方法需要如深度线索、偏振度等额外信息,需要通过用户交互或各种相机定位操作提供。因此,这些方法不适用于实时视觉应用。基于多幅图像的方法也需要从不同偏振度、不同天气场景下拍摄的多幅图像。这些方法需要额外的硬件或其他资源。因此,这些方法比SID计算代价更高,SID算法具有很强的实用性,胜任实时性要求高的场景。